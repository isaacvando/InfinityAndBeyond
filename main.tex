\documentclass[compress,17pt]{beamer}

\usepackage{tikz-cd}
\usepackage{tikz}


\newenvironment{diagram}{\begin{center}\begin{tikzcd}[ampersand replacement=\&, column sep=small, font=\small]}{\end{tikzcd}\end{center}}

\newcommand{\N}{\mathbb{N}}
\newcommand{\Z}{\mathbb{Z}}
\newcommand{\Q}{\mathbb{Q}}
\newcommand{\R}{\mathbb{R}}
\newcommand{\C}{\mathbb{C}}

\usetheme{Arguelles}

\title{Infinity and Beyond}
% \subtitle{Reaching}
\author{Isaac Van Doren}
\date{\today}

\beamertemplatenavigationsymbolsempty

% \tikzcdset{every label/.append style = {font = \small}}


\begin{document}


% Intuition
% We have some intuition about infinity but it remains mysterious
% Math can clarify these issues

\frame{\titlepage}


% Sets are the vehicle we can use to make infinity precise.
% finite sets
\begin{frame}
  \frametitle{Sets}
  \{\} \\
  \{\tikz\draw[red,fill=red] (0,0) circle (.5ex);,\tikz\draw[blue,fill=blue] (0,0) circle (.5ex);,\tikz\draw[green,fill=green] (0,0) circle (.5ex);\} \\
  \{1,2,3,4\} \\
  \{890, ``foo'', $\pi$\} \\
\end{frame}

% infinite sets
\begin{frame}
  \frametitle{Infinite Sets}
  $\N = \{1,2,3,4,5,6,7,8,9,10,\dots\} $
  $\Z = \{0,-1,1,-2,2,-3,3,-4,4,-5,5,\dots\}$
  $\Q = \{0,1,2,\frac12, 3, 4, \frac32, \frac23, \frac14, \frac15, 5, 6, \frac52, \dots\}$
\end{frame}
\begin{frame}
  \frametitle{Infinite Sets}
  \{ every possible book \} \\
  \{ every possible book that starts with ``supercalifragilisticexpialidocious''\} \\
  \{ every point on a circle \} \\
  \{ every possible board game \} \\
  \{ every color \} \\
  \{ every triangle \}
\end{frame}

% How do we say two sets are the same size?
\begin{frame}{Size}
  How can we tell if two sets are the same size? \pause
  Just count them!
\end{frame}

\begin{frame}{Size}
  \{Porto, Columbia, Nashville, Denver\} \\
  \{ cherries, grapefruit, apples, kiwis \}
\end{frame}

\begin{frame}{Size}
  \begin{diagram}
    1 \arrow[d, leftrightarrow] \& 2 \arrow[d, leftrightarrow] \& 3 \arrow[d, leftrightarrow] \& 4 \arrow[d, leftrightarrow] \\
    \{ Porto, \& Columbia, \& Nashville, \& Denver \}
  \end{diagram}

  \begin{diagram}
    1 \arrow[d, leftrightarrow] \& 2 \arrow[d, leftrightarrow] \& 3 \arrow[d, leftrightarrow] \& 4 \arrow[d, leftrightarrow] \\
    \{ cherries, \& grapefruit, \& apples, \& kiwis \}
  \end{diagram} \pause
\end{frame}

\begin{frame}{Size}
  4 = 4 \\ \pause
  \# cities = \# fruits
\end{frame}

% \begin{frame}{Size}
%   \begin{diagram}
%     \{ 1, \arrow[d, leftrightarrow] \& 2, \arrow[d, leftrightarrow] \& 3, \arrow[d, leftrightarrow] \& 4 \arrow[d, leftrightarrow] \} \\
%     \{ Angela, \& Don, \& Sonia, \& Jordan \}
%   \end{diagram}
% \end{frame}


\begin{frame}
  \begin{align*}
    \{1,2,3,4,5,6,\dots\} \\
    \{1,2,3,4,5,6,\dots\}
  \end{align*}
\end{frame}

\begin{frame}
  \begin{diagram}
    \{ 1, \arrow[d, leftrightarrow] \& 2, \arrow[d, leftrightarrow] \& 3, \arrow[d, leftrightarrow] \& 4, \arrow[d, leftrightarrow] \& \dots \} \\
    \{ 1, \& 2, \& 3, \& 4, \& \dots \}
  \end{diagram}
\end{frame}




% Outline

% intuitions about infinity
% - it is the biggest thing, and yet inf + 1 = inf?

% sets

% What it means to say two things are the same size in the finite case
% - counting to a natural number, equivalent to pairing up each item

% grouping items 

% math is about generalizing ideas. We can't count up to a natural number anymore but we can still pair items up with each other. Negative numbers are a generalizing.
% the naturals are in bijection with the regular polygons. So they have the same size. Also in bij with self

% BUT, counter to our intuition, subsets can be the same size. 
% Show example in the finite case where this is obviously not true
% Show that finite subsets of an infinite set are not the same size
% Show that the naturals are in bijection with the even numbers, and that it can go as far as we want, like the perfect squares, or further
% We can take this further to N x N. Or Even N x N x N, or even more! Or show Q

% BUT, it stops at a point. the naturals are not in bij with the binary sequences
% - cantor space is a whole new size of infinity, the same as the real numbers

% We can go further. Introduce powerset.
%  Maybe mention functions R -> R are \aleph_2
% no biggest infinity

% Continuum hypothesis

% This is only the beginning. There is infinty beyond this infinity chain! Mention large cardinals. 

% Close with HOW BIG infinity is. How amazing it is we can work with it. How mysterious it is. Maybe some follow up questinos that couldn't be touched on.





\end{document}


% Infinity and What Lies Beyond
% To Infinity and Beyond
% Grasping Infinity
% Comprehending Infinity
% Walking Past Infinity
% Going Beyond Infinity
% What Lies Beyond Infinity
% What Comes After Infinity
% Infinity and Beyond


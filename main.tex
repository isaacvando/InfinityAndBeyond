\documentclass[compress,17pt]{beamer}

\usepackage{tikz-cd}
\usepackage{tikz}



\newenvironment{diagram}{\begin{tikzcd}[ampersand replacement=\&, column sep=small, font=\small]}{\end{tikzcd}}

\newcommand{\N}{\mathbb{N}}
\newcommand{\Z}{\mathbb{Z}}
\newcommand{\Q}{\mathbb{Q}}
\newcommand{\R}{\mathbb{R}}
\newcommand{\C}{\mathbb{C}}

\usepackage[normalem]{ulem}
\newcommand\hl{\bgroup\markoverwith
    {\textcolor{yellow}{\rule[-.5ex]{.1pt}{2.5ex}}}\ULon}

\newcommand{\polygon}[1]{
  \begin{tikzpicture}
    \def\n{#1} % Number of sides
    \def\radius{0.3cm} % Radius of the polygon
    \pgfmathsetmacro\angle{360/\n}
    \foreach \i in {1,...,\n} {
      \draw (360/\n*\i:\radius) -- (360/\n*\i+\angle:\radius);
    }
  \end{tikzpicture}
}

\usetheme{Arguelles}

\title{Infinity and Beyond}
\author{Isaac Van Doren}
\date{\today}

\beamertemplatenavigationsymbolsempty

% \tikzcdset{every label/.append style = {font = \small}}


\begin{document}


% Intuition
% We have some intuition about infinity but it remains mysterious
% Math can clarify these issues

\frame{\titlepage}


% Sets are the vehicle we can use to make infinity precise.
% finite sets
\begin{frame}
  \frametitle{Finite Sets} \pause
  \{\} \\ \pause
  \{\tikz\draw[red,fill=red] (0,0) circle (.5ex);,\tikz\draw[blue,fill=blue] (0,0) circle (.5ex);,\tikz\draw[green,fill=green] (0,0) circle (.5ex);\} \\ \pause

  \{1,2,3,4\} \\ \pause
  \{890, ``foo'', $\pi$\} \\
\end{frame}

% infinite sets
\begin{frame}
  \frametitle{Infinite Sets} \pause
  $\N = \{1,2,3,4,5,6,7,8,9,10,\dots\} $ \pause
  $\Z = \{0,-1,1,-2,2,-3,3,-4,4,-5,5,\dots\}$ \pause
  $\Q = \{0,1,2,\frac12, 3, 4, \frac32, \frac23, \frac14, \frac15, 5, 6, \frac52, \dots\}$
\end{frame}
\begin{frame}
  \frametitle{Infinite Sets} \pause
  \{ every possible book \} \\ \pause
  \{ every possible book that starts with ``supercalifragilisticexpialidocious''\} \\ \pause
  \{ every point on a circle \} \\ \pause
  \{ every possible board game \} \\ \pause
  \{ every color \} \\ \pause
  \{ every triangle \}
\end{frame}

\begin{frame}{Subsets} \pause
  $\{1,2,3\} \subseteq \{1,2,3,4,5\}$ \pause
  $\{700,543,\frac12\} \not\subseteq \{1,3,5,700\}$ \pause

  $\{\} \subseteq \{0,1\}$
\end{frame}

% How do we say two sets are the same size?
\begin{frame}{Size}
  \centering
  How can we tell if two sets are the same size? \pause
  Just count them!
\end{frame}

\begin{frame}{Size}
  \centering
  \{Porto, Columbia, Nashville, Denver\} \\
  \{ cherries, grapefruit, apples, kiwis \}
\end{frame}

\begin{frame}{Size}
  \centering
  \begin{diagram}
    1 \arrow[d, leftrightarrow] \& 2 \arrow[d, leftrightarrow] \& 3 \arrow[d, leftrightarrow] \& 4 \arrow[d, leftrightarrow]  \\
    \{ Porto, \& Columbia, \& Nashville, \& Denver \}
  \end{diagram}

  \begin{diagram}
    1 \arrow[d, leftrightarrow] \& 2 \arrow[d, leftrightarrow] \& 3 \arrow[d, leftrightarrow] \& 4 \arrow[d, leftrightarrow] \\
    \{ cherries, \& grapefruit, \& apples, \& kiwis \}
  \end{diagram}
\end{frame}

\begin{frame}{Size}
  \centering
  4 = 4 \\ \pause
  \# cities = \# fruits
\end{frame}

\begin{frame}{Size}
  \centering
  \begin{diagram}
    \{ 1, \arrow[d, leftrightarrow] \& 2, \arrow[d, leftrightarrow] \& 3, \arrow[d, leftrightarrow] \& 4 \arrow[d, leftrightarrow] \} \\
    \{ Porto, \& Columbia, \& Nashville, \& Denver \}
  \end{diagram}

  \begin{diagram}
    \{1, \arrow[d, leftrightarrow] \& 2, \arrow[d, leftrightarrow] \& 3, \arrow[d, leftrightarrow] \& 4 \arrow[d, leftrightarrow] \} \\
    \{ cherries, \& grapefruit, \& apples, \& kiwis \}
  \end{diagram}
\end{frame}

\begin{frame}{Size}
  \centering
  \begin{diagram}
    \{Porto, \arrow[d, leftrightarrow] \& Columbia, \arrow[d, leftrightarrow] \& Nashville, \arrow[d, leftrightarrow] \& Denver \arrow[d, leftrightarrow] \} \\
    \{ cherries, \& grapefruit, \& apples, \& kiwis \}
  \end{diagram}
\end{frame}

% \begin{frame}{Size}
%   \begin{diagram}
%     \{ 1, \arrow[d, leftrightarrow] \& 2, \arrow[d, leftrightarrow] \& 3, \arrow[d, leftrightarrow] \& 4 \arrow[d, leftrightarrow] \} \\
%     \{ Angela, \& Don, \& Sonia, \& Jordan \}
%   \end{diagram}
% \end{frame}


% \begin{frame}
%   \begin{align*}
%     \{1,2,3,4,5,6,\dots\} \\
%     \{1,2,3,4,5,6,\dots\}
%   \end{align*}
% \end{frame}

\begin{frame}{Generalization}
  \pause
  \begin{diagram}
    \{ 1, \arrow[d, leftrightarrow] \& 2, \arrow[d, leftrightarrow] \& 3, \arrow[d, leftrightarrow] \& 4, \arrow[d, leftrightarrow] \& 5, \arrow[d, leftrightarrow] \& \dots \} \\
    \{ ., \& \polygon{2}, \& \polygon{3}, \& \polygon{4}, \& \polygon{5}, \& \dots \}
  \end{diagram}
\end{frame}


% \begin{frame}[standout]
%   \centering\LARGE
%   Surprise \#1
% \end{frame}

\begin{frame}{Surprise \#1 - Subsets}
  \centering \pause
  $\{1,2\} \subseteq \{1,2,3,4\}$ \pause

  \begin{diagram}
    \{ 1, \arrow[d, leftrightarrow] \& 2, \arrow[d, leftrightarrow] \& 3,  \& 4, \} \\
    \{ 1, \& 2 \}
  \end{diagram}
\end{frame}

\begin{frame}{Surprise \#1 - Subsets}
  \centering
  $\{3,4,5,6,7,\dots\} \subseteq \{1,2,3,4,5,\dots\}$ \pause

  \begin{diagram}
    \{ 1, \arrow[d, leftrightarrow] \& 2, \arrow[d, leftrightarrow] \& 3, \arrow[d, leftrightarrow] \& 4, \arrow[d, leftrightarrow] \& 5, \arrow[d, leftrightarrow] \& \dots \} \\
    \{ 3, \& 4, \& 5, \& 6, \& 7, \& \dots \}
  \end{diagram}
\end{frame}

\begin{frame}{Surprise \#1 - Subsets}
  \centering
  $\{2,4,6,8,10,\dots\} \subseteq \{1,2,3,4,5,\dots\}$ \pause

  \begin{diagram}
    \{ 1, \arrow[d, leftrightarrow] \& 2, \arrow[d, leftrightarrow] \& 3, \arrow[d, leftrightarrow] \& 4, \arrow[d, leftrightarrow] \& 5, \arrow[d, leftrightarrow] \& \dots \} \\
    \{ 2, \& 4, \& 6, \& 8, \& 10, \& \dots \}
  \end{diagram}
\end{frame}

\begin{frame}{Surprise \#1 - Subsets}
  \centering
  $\{1,4,9,16,25,\dots\} \subseteq \{1,2,3,4,5,\dots\}$ \pause

  \begin{diagram}
    \{ 1, \arrow[d, leftrightarrow] \& 2, \arrow[d, leftrightarrow] \& 3, \arrow[d, leftrightarrow] \& 4, \arrow[d, leftrightarrow] \& 5, \arrow[d, leftrightarrow] \& \dots \} \\
    \{ 1, \& 4, \& 9, \& 16, \& 25, \& \dots \}
  \end{diagram}
\end{frame}


\begin{frame}{Surprise \#1 - Subsets}
  \centering
  $\{1,32,243,1024,3125,\dots\} \subseteq \{1,2,3,4,5,\dots\}$ \pause

  \begin{diagram}
    \{ 1, \arrow[d, leftrightarrow] \& 2, \arrow[d, leftrightarrow] \& 3, \arrow[d, leftrightarrow] \& 4, \arrow[d, leftrightarrow] \& 5, \arrow[d, leftrightarrow] \& \dots \} \\
    \{ 1, \& 32, \& 243, \& 1024, \& 3125, \& \dots \}
  \end{diagram}
\end{frame}


\begin{frame}{Surprise \#1 - Subsets}
  \centering \pause
  So infinity is infinity is infinity then... right? \pause
  No!
\end{frame}


\begin{frame}[standout]
  There are different sized infinities!
\end{frame}

\begin{frame}{Surprise \#2 - Sizes}
  The Cantor Set \\ \pause
  $2^\omega$ = \{all infinite binary sequences\} \\
  = \{00000000\dots, 111111111\dots, 10101010\dots, 00110100\dots, 11111010\dots, \dots\}\\  \pause
  \vphantom{text}

  Can we count it?
\end{frame}

\begin{frame}[standout]
  \large
  Cantor's Diagonal Argument
\end{frame}

\begin{frame}
  \centering
  \begin{tabular}{ | c | c c c c c c c c c }
    \hline
    &  \\
    \hline

    1 & 1 & 0 & 0 & 1 & 0 & 0 & 0 & $\cdots$\\
    2 & 0 & 0 & 0 & 1 & 1 & 1 & 1 & $\cdots$\\
    3 & 0 & 1 & 1 & 1 & 0 & 0 & 1 & $\cdots$\\
    4 & 0 & 0 & 1 & 1 & 0 & 0 & 0 & $\cdots$\\
    5 & 1 & 1 & 1 & 0 & 1 & 0 & 0 & $\cdots$\\
    6 & 0 & 1 & 0 & 1 & 0 & 1 & 0 & $\cdots$\\
    7 & 0 & 0 & 0 & 0 & 1 & 0 & 0 & $\cdots$\\
    $\vdots$ & $\vdots$ & $\vdots$ & $\vdots$ & $\vdots$ & $\vdots$ & $\vdots$ & $\vdots$ & \\
  \end{tabular}
\end{frame}

\begin{frame}
  \centering
  \begin{tabular}{ | c | c c c c c c c c c }
    \hline
    &  \hl 0 \\
    \hline

    1 & \hl 1 & 0 & 0 & 1 & 0 & 0 & 0 & $\cdots$\\
    2 & 0 & 0 & 0 & 1 & 1 & 1 & 1 & $\cdots$\\
    3 & 0 & 1 & 1 & 1 & 0 & 0 & 1 & $\cdots$\\
    4 & 0 & 0 & 1 & 1 & 0 & 0 & 0 & $\cdots$\\
    5 & 1 & 1 & 1 & 0 & 1 & 0 & 0 & $\cdots$\\
    6 & 0 & 1 & 0 & 1 & 0 & 1 & 0 & $\cdots$\\
    7 & 0 & 0 & 0 & 0 & 1 & 0 & 0 & $\cdots$\\
    $\vdots$ & $\vdots$ & $\vdots$ & $\vdots$ & $\vdots$ & $\vdots$ & $\vdots$ & $\vdots$ & \\
  \end{tabular}
\end{frame}

\begin{frame}
  \centering
  \begin{tabular}{ | c | c c c c c c c c c }
    \hline
    &  0 & \hl 1 \\
    \hline

    1 & 1 & 0 & 0 & 1 & 0 & 0 & 0 & $\cdots$\\
    2 & 0 & \hl 0 & 0 & 1 & 1 & 1 & 1 & $\cdots$\\
    3 & 0 & 1 & 1 & 1 & 0 & 0 & 1 & $\cdots$\\
    4 & 0 & 0 & 1 & 1 & 0 & 0 & 0 & $\cdots$\\
    5 & 1 & 1 & 1 & 0 & 1 & 0 & 0 & $\cdots$\\
    6 & 0 & 1 & 0 & 1 & 0 & 1 & 0 & $\cdots$\\
    7 & 0 & 0 & 0 & 0 & 1 & 0 & 0 & $\cdots$\\
    $\vdots$ & $\vdots$ & $\vdots$ & $\vdots$ & $\vdots$ & $\vdots$ & $\vdots$ & $\vdots$ & \\
  \end{tabular}
\end{frame}

\begin{frame}
  \centering
  \begin{tabular}{ | c | c c c c c c c c c }
    \hline
    &  0 & 1 & \hl 0 \\
    \hline

    1 & 1 & 0 & 0 & 1 & 0 & 0 & 0 & $\cdots$\\
    2 & 0 & 0 & 0 & 1 & 1 & 1 & 1 & $\cdots$\\
    3 & 0 & 1 & \hl 1 & 1 & 0 & 0 & 1 & $\cdots$\\
    4 & 0 & 0 & 1 & 1 & 0 & 0 & 0 & $\cdots$\\
    5 & 1 & 1 & 1 & 0 & 1 & 0 & 0 & $\cdots$\\
    6 & 0 & 1 & 0 & 1 & 0 & 1 & 0 & $\cdots$\\
    7 & 0 & 0 & 0 & 0 & 1 & 0 & 0 & $\cdots$\\
    $\vdots$ & $\vdots$ & $\vdots$ & $\vdots$ & $\vdots$ & $\vdots$ & $\vdots$ & $\vdots$ & \\
  \end{tabular}
\end{frame}

\begin{frame}
  \centering
  \begin{tabular}{ | c | c c c c c c c c c }
    \hline
    &  0 & 1 & 0 & \hl 0 \\
    \hline

    1 & 1 & 0 & 0 & 1 & 0 & 0 & 0 & $\cdots$\\
    2 & 0 & 0 & 0 & 1 & 1 & 1 & 1 & $\cdots$\\
    3 & 0 & 1 & 1 & 1 & 0 & 0 & 1 & $\cdots$\\
    4 & 0 & 0 & 1 & \hl 1 & 0 & 0 & 0 & $\cdots$\\
    5 & 1 & 1 & 1 & 0 & 1 & 0 & 0 & $\cdots$\\
    6 & 0 & 1 & 0 & 1 & 0 & 1 & 0 & $\cdots$\\
    7 & 0 & 0 & 0 & 0 & 1 & 0 & 0 & $\cdots$\\
    $\vdots$ & $\vdots$ & $\vdots$ & $\vdots$ & $\vdots$ & $\vdots$ & $\vdots$ & $\vdots$ & \\
  \end{tabular}
\end{frame}

\begin{frame}
  \centering
  \begin{tabular}{ | c | c c c c c c c c c }
    \hline
    &  0 & 1 & 0 & 0 & \hl 0 \\
    \hline

    1 & 1 & 0 & 0 & 1 & 0 & 0 & 0 & $\cdots$\\
    2 & 0 & 0 & 0 & 1 & 1 & 1 & 1 & $\cdots$\\
    3 & 0 & 1 & 1 & 1 & 0 & 0 & 1 & $\cdots$\\
    4 & 0 & 0 & 1 & 1 & 0 & 0 & 0 & $\cdots$\\
    5 & 1 & 1 & 1 & 0 & \hl 1 & 0 & 0 & $\cdots$\\
    6 & 0 & 1 & 0 & 1 & 0 & 1 & 0 & $\cdots$\\
    7 & 0 & 0 & 0 & 0 & 1 & 0 & 0 & $\cdots$\\
    $\vdots$ & $\vdots$ & $\vdots$ & $\vdots$ & $\vdots$ & $\vdots$ & $\vdots$ & $\vdots$ & \\
  \end{tabular}
\end{frame}


\begin{frame}
  \centering
  \begin{tabular}{ | c | c c c c c c c c c }
    \hline
    &  0 & 1 & 0 & 0 & 0 & \hl 0\\
    \hline

    1 & 1 & 0 & 0 & 1 & 0 & 0 & 0 & $\cdots$\\
    2 & 0 & 0 & 0 & 1 & 1 & 1 & 1 & $\cdots$\\
    3 & 0 & 1 & 1 & 1 & 0 & 0 & 1 & $\cdots$\\
    4 & 0 & 0 & 1 & 1 & 0 & 0 & 0 & $\cdots$\\
    5 & 1 & 1 & 1 & 0 & 1 & 0 & 0 & $\cdots$\\
    6 & 0 & 1 & 0 & 1 & 0 & \hl 1 & 0 & $\cdots$\\
    7 & 0 & 0 & 0 & 0 & 1 & 0 & 0 & $\cdots$\\
    $\vdots$ & $\vdots$ & $\vdots$ & $\vdots$ & $\vdots$ & $\vdots$ & $\vdots$ & $\vdots$ & \\
  \end{tabular}
\end{frame}

\begin{frame}
  \centering
  \begin{tabular}{ | c | c c c c c c c c c }
    \hline
    & 0 & 1 & 0 & 0 & 0 & 0 & \hl 1 \\
    \hline

    1 & 1 & 0 & 0 & 1 & 0 & 0 & 0 & $\cdots$\\
    2 & 0 & 0 & 0 & 1 & 1 & 1 & 1 & $\cdots$\\
    3 & 0 & 1 & 1 & 1 & 0 & 0 & 1 & $\cdots$\\
    4 & 0 & 0 & 1 & 1 & 0 & 0 & 0 & $\cdots$\\
    5 & 1 & 1 & 1 & 0 & 1 & 0 & 0 & $\cdots$\\
    6 & 0 & 1 & 0 & 1 & 0 & 1 & 0 & $\cdots$\\
    7 & 0 & 0 & 0 & 0 & 1 & 0 & \hl 0 & $\cdots$\\
    $\vdots$ & $\vdots$ & $\vdots$ & $\vdots$ & $\vdots$ & $\vdots$ & $\vdots$ & $\vdots$ & \\
  \end{tabular}
\end{frame}


\begin{frame}
  \centering
  \begin{tabular}{ | c | c c c c c c c c c }
    \hline
    & \hl 0 & \hl 1 & \hl 0 & \hl 0 & \hl 0 & \hl 0 & \hl 1 & \dots \\
      \hline
    1 & \hl1 & 0 & 0 & 1 & 0 & 0 & 0 & $\cdots$\\
    2 & 0 & \hl0 & 0 & 1 & 1 & 1 & 1 & $\cdots$\\
    3 & 0 & 1 & \hl1 & 1 & 0 & 0 & 1 & $\cdots$\\
    4 & 0 & 0 & 1 & \hl1 & 0 & 0 & 0 & $\cdots$\\
    5 & 1 & 1 & 1 & 0 & \hl1 & 0 & 0 & $\cdots$\\
    6 & 0 & 1 & 0 & 1 & 0 & \hl1 & 0 & $\cdots$\\
    7 & 0 & 0 & 0 & 0 & 1 & 0 & \hl 0 & $\cdots$\\
    $\vdots$ & $\vdots$ & $\vdots$ & $\vdots$ & $\vdots$ & $\vdots$ & $\vdots$ & $\vdots$ & \\
  \end{tabular}
\end{frame}

\begin{frame}{Surprise \#2 - Sizes}
  \centering
  We can't pair up each element of $\N$ with $2^\omega$.\\ \pause
  $2^\omega > \N$!
\end{frame}

\begin{frame}{Infinitely Many Infinities}
  There are infinitely many sizes of infinities!
\end{frame}

\begin{frame}{Infinitely Many Infinities}
  $\mathcal{P}(S) = \{ \text{all subsets of } S \}$ \\ \pause
  $\mathcal{P}(\{1,2,3\}) = \{\{\}, \{1\}, \{2\}, \{3\}, \{1,2\}, \{1,3\}, \{2,3\}, \{1,2,3\}\}$
\end{frame}

\begin{frame}{Infinitely Many Infinities}
  \small
  $\N < \mathcal{P}(\N) < \mathcal{P}(\mathcal{P}(\N)) < \mathcal{P}(\mathcal{P}(\mathcal{P}(\N))) < \cdots$ \pause \\
  $\aleph_0 < \aleph_1 < \aleph_2 < \aleph_3 < \cdots $
\end{frame}

\begin{frame}{Infinitely Many Infinities}
  \centering
  It keeps on going! \pause \\
  $\aleph_\omega$
\end{frame}

\begin{frame}{Further Exploration}
  \begin{itemize}
    \item How To Count Past Infinity - Vsauce
    \item The Mystery of the Aleph - Amir D. Aczel
    \item Infinitely More - Joel David Hamkins
  \end{itemize}
\end{frame}

% Outline

% intuitions about infinity
% - it is the biggest thing, and yet inf + 1 = inf?

% sets

% What it means to say two things are the same size in the finite case
% - counting to a natural number, equivalent to pairing up each item

% grouping items 

% math is about generalizing ideas. We can't count up to a natural number anymore but we can still pair items up with each other. Negative numbers are a generalizing.
% the naturals are in bijection with the regular polygons. So they have the same size. Also in bij with self

% BUT, counter to our intuition, subsets can be the same size. 
% Show example in the finite case where this is obviously not true
% Show that finite subsets of an infinite set are not the same size
% Show that the naturals are in bijection with the even numbers, and that it can go as far as we want, like the perfect squares, or further
% We can take this further to N x N. Or Even N x N x N, or even more! Or show Q

% BUT, it stops at a point. the naturals are not in bij with the binary sequences
% - cantor space is a whole new size of infinity, the same as the real numbers

% We can go further. Introduce powerset.
%  Maybe mention functions R -> R are \aleph_2
% no biggest infinity

% Continuum hypothesis

% This is only the beginning. There is infinty beyond this infinity chain! Mention large cardinals. 

% Close with HOW BIG infinity is. How amazing it is we can work with it. How mysterious it is. Maybe some follow up questinos that couldn't be touched on.





\end{document}
